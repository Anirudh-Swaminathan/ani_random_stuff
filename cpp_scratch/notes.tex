\documentclass[11pt]{book}
\usepackage[sc,osf]{mathpazo}   % With old-style figures and real smallcaps.
\linespread{1.025}              % Palatino leads a little more leading
% Euler for math and numbers
\usepackage[euler-digits,small]{eulervm}
%\documentclass[10pt]{llncs}
%\usepackage{llncsdoc}
\usepackage{amsmath,amssymb}
\usepackage{graphicx}
\usepackage{makeidx}
\usepackage{algpseudocode}
\usepackage{algorithm}
\usepackage{listing}
\usepackage{comment}
\usepackage{physics}
% look for package for quantum computing!
\evensidemargin=0.20in
\oddsidemargin=0.20in
\topmargin=0.2in
%\headheight=0.0in
%\headsep=0.0in
%\setlength{\parskip}{0mm}
%\setlength{\parindent}{4mm}
\setlength{\textwidth}{6.4in}
\setlength{\textheight}{8.5in}
%\leftmargin -2in
%\setlength{\rightmargin}{-2in}
%\usepackage{epsf}
%\usepackage{url}



\usepackage{booktabs}   %% For formal tables:
                        %% http://ctan.org/pkg/booktabs
\usepackage{subcaption} %% For complex figures with subfigures/subcaptions
                        %% http://ctan.org/pkg/subcaption
\usepackage{enumitem}
\usepackage{minted}
%\newminted{fortran}{fontsize=\footnotesize}

\usepackage{xargs}
\usepackage[colorinlistoftodos,prependcaption,textsize=tiny]{todonotes}

\usepackage{hyperref}
\hypersetup{
    colorlinks,
}

\usepackage{epsfig}
\usepackage{tabularx}
\usepackage{latexsym}
\newtheorem{lemma}{Lemma}
\newtheorem{observation}{Observation}
\newtheorem{proof}{Proof}


\def\qed{$\Box$}
\def\proof{\textit{Proof. }}
\newtheorem{corollary}{Corollary}
\newtheorem{theorem}{Theorem}
% \DeclareMathOperator{\tr}{trace}

\title{learncpp.com notes | summary + completion notes}
\author{Anirudh Swaminathan}
\date{May 23, 2022}

\begin{document}

% qubit
\newcommand{\qb}[1]{\ensuremath{|#1\rangle}}
% \newcommand{\braket}[2]{\ensuremath{\langle#1~\vert~#2\rangle}}
%\newcommand{\ket}[2]{\ensuremath{\langle#1~\vert~#2\rangle}}
\newcommand{\Z}{\ensuremath{\mathbb{Z}}}
\newcommand{\R}{\ensuremath{\mathbb{R}}}
\newcommand{\C}{\ensuremath{\mathbb{C}}}
\renewcommand{\H}{\ensuremath{\mathbb{H}}}
\newcommand{\tensor}{\ensuremath{\otimes}}
% \newcommand{innerprod}[2]{\ensuremath{\bra{#1}{\ket{#2}}}}
\newcommand{\E}[1]{\mathbb{E}\left[ #1 \right]}
\renewcommand{\l}{\left}
\renewcommand{\r}{\right}
\newcommand{\xor}{\ensuremath{\oplus}}
%\newcommand{\tensor}{\ensuremath{\otimes}}

\maketitle
\tableofcontents
\setcounter{chapter}{-1}
\chapter{Tutorial Overview}

Topics:
\begin{itemize}
    \item Intro
    \item Machine Language, Assembly Language and High Level Language
    \item Intro to C++
    \item 6 steps of development
    \item CodeBlocks installation
    \item Building projects and compiling code
    \item IDE build configurations
    \item 13 common C++ problems
\end{itemize}

\chapter{C++ Basics}

Topics:
\begin{itemize}
    \item Statements, Expressions, functions and libraries. Syntax Errors
    \item Comments. 4 reasons to write comments. Good comments
    \item Varaibles, definition. l-values and r-values. Initialization
    vs. Assignment. Undefined behaviour
    \item cout, endl, cin, output(<<) and input(>>) operators
    \item functions, return values and types, reusing functions,
    function call precedence over (<<), pointers to function names
    \item Functions
    \begin{itemize}
        \item function parameters and arguments, pass-by-value
        \item Usefulness of functions - Organization, reusability, testing, 
        extensibility, abstraction. Refactoring meaning
        \item Keywords and naming identifiers, good naming conventions
        \item Local scope
    \end{itemize}
    \item Literals, operands, operators -> unary, binary, and ternary
    \item Whitespaces, basic formatting. 6 rules to follow
    \item Forward declarations of functions. Declaration vs definition. ODR. Function prototypes.
    \item Multiple files for a single program.
    \begin{itemize}
        \item Naming conflicts, namespaces and the std namespace
    \end{itemize}
    \item Headers, their purposes. Header guards. Declaring vs defining
    functions in headers. Handling different directories. Best pracices.
    \item Preprocessors and Headers
    \begin{itemize}
        \item Preprocessors. Macro defines -> function defines and object defines
        Object defines with and without substitution text. Conditional compilation. Scope of defines -> Only within a single file.
        \item Header guards to remove duplicate definitions of functions. Limitations of header guards. Multiple declarations allowed, not 
        multiple definitions. \#pragma usage.
        \item 8 steps for designing any program. Some advice on writing programs
    \end{itemize}
    \item Syntax and semantic errors. Debugger. Stepping -> into, over, out. Run to cursor. Run/continue/go. Breakpoints.
    \item Comprehensive quiz on Chapter 1 concepts
\end{itemize}

\chapter{C++ Basics: Functions and Files}

Topics:
\begin{itemize}
    \item Basics of addressing memory.
    \item Void
    \item Variable size and sizeof operator
    \item Integers
    \begin{itemize}
        \item Introduction
        \item Fixed-width integers
    \end{itemize}
\end{itemize}

\section{Basics of addressing memory}
Fundamental data types - bool, char, float, int and void.
Variable definition and initialization. Copy initializaton, direct initialization and uniform initialization.
Good practices:-
\begin{itemize}
    \item Direct initialization is preferred over copy initialization.
    \item Uniform initialization is preferred for newer versions of the C++
    compiler.
    \item Always initialize fundamental variables or assign a value to them as
    soon as possible after their definition.
    \item Don't define multiple variables in the same line if initializing any
    of them.
    \item Define variables as close to their first use as far as possible.
\end{itemize}

\section{Void}
Usage. Empty parameter list(implicit void) instead of specfying void.
Parameter, return value and pointers

\section{Variable size and the sizeof operator}
\textbackslash t is a tab character

\section{Integers}

\subsection{Introduction}
5 fundamental integer types -> char, short, int, long, long long.
Important! C++ only guarantees types will have some minimum size, and not a specific size. signed and unsigned integers.
n-bit signed range -> \[ -2^(n-1) to +(2^(n-1))-1 \]
n-bit unsigned range -> \[ 0 to +2^(n)-1 \]
Default is signed. Favour signed integers to unsigned integers
Integer OVerflow. Do NOT depend on the results of overflow for programs
Integer Division fundamentals.

\subsection{Fixed-width integers}
C99 stdint.h C11 cstdint. Major types are:- \\
\begin{tabular}{|c|c|}
    \hline
    \textbf{Type} & \textbf{Type Size} \\
    \hline
    int8\_t & 1 byte unsigned \\
    uint8\_t & 1 byte unsigned \\
    int16\_t & 2 byte signed \\
    uint16\_t & 2 byte unsigned \\
    int32\_t & 4 byte signed \\
    uint32\_t & 4 byte unsigned \\
    int64\_t & 8 byte signed \\
    uint64\_t & 8 byte unsigned \\
    \hline
\end{tabular}
\\
All of these types defined inside the STD namespace.
IMP - Avoid int8\_t and uint8\_t as much as possible. If used, they are
usually used as characters
Fast and Least types. For example, int\_fast32\_t, int\_least32\_t.
$6$ integers best practices:-
\begin{itemize}
    \item Prefer int when size of integer doesn't matter
    \item Use int\_fast\#\_t for performance. (\# -> $8, 16, 32, 64$)
    \item Use int\_least\#\_t for memory. (\# -> $8, 16, 32, 64$)
    \item Use unsigned only when there is a compelling reason
    \item Avoid compiler defined versions of fixed width types
    \item Don't mix signed and unsigned types unless I want my code to go ballistic!!
\end{itemize}

One important point is that C++ will freely convert between signed and
unsigned numbers, but it won’t do any range checking to make sure you
don’t overflow your type. Hence there's no good way to guard against
mixing of signed and unsigned integers

\end{document}